\section{The VCVio Library}\label{sec:vcvio}

This library provides a formal framework for reasoning about computations that make \emph{oracle
queries}. Many cryptographic primitives and interactive protocols use oracles to model (or simulate)
external functionality such as random responses, coin flips, or more structured queries. The VCVio
library "lifts" these ideas into a setting where both the abstract specification and concrete
simulation of oracles may be studied, and their probabilistic behavior analyzed.

The main ingredients of the library are as follows:

\begin{definition}[Specification of Oracles]
    \label{def:oracle_spec}
    An oracle specification describes a collection of available oracles, each with its own input and output types. 
    Formally, it's given by an indexed family where each oracle is specified by:
    \begin{itemize}
        \item A domain type (what inputs it accepts)
        \item A range type (what outputs it can produce)
    \end{itemize}
    The indexing allows for potentially infinite collections of oracles, and the specification itself 
    is agnostic to how the oracles actually behave - it just describes their interfaces.
    \lean{OracleSpec}
    \leanok
\end{definition}

Some examples of oracle specifications (and their intended behavior) are as follows:
\begin{itemize}
    \item \verb|emptySpec|: Represents an empty set of oracles
    \item \verb|singletonSpec|: Represents a single oracle available on a singleton index
    \item \verb|coinSpec|: A coin flipping oracle that produces a random Boolean value
    \item \verb|unifSpec|: A family of oracles that for every natural number $n \in \mathbb{N}$ chooses uniformly from the set $\{0, \ldots, n\}$.
\end{itemize}
    \lean{emptySpec, singletonSpec, coinSpec, unifSpec}

We often require extra properties on the domains and ranges of oracles. For example, we may require that the domains and ranges come equipped with decidable equality \lean{OracleSpec.DecidableEq}
or finiteness properties \lean{OracleSpec.FiniteRange}.

\begin{definition}[Oracle Computation]
    \label{def:oracle_computation}
    An oracle computation represents a program that can make oracle queries. It can:
    \begin{itemize}
        \item Return a pure value without making any queries (via \texttt{pure})
        \item Make an oracle query and continue with the response (via \texttt{queryBind})
        \item Signal failure (via \texttt{failure})
    \end{itemize}
    The formal implementation uses a free monad on the inductive type of oracle queries \lean{OracleSpec.OracleQuery} wrapped in an option monad transformer (i.e. \verb|OptionT(FreeMonad(OracleQuery spec))|).
    \lean{OracleComp}
    \uses{def:oracle_spec}
\end{definition}

\begin{definition}[Handling Oracle Queries]
    \label{def:handling_oracle_queries}
    To actually run oracle computations, we need a way to handle (or implement) the oracle queries.
    An oracle implementation consists a mapping from oracle queries to values in another monad. Depending on the monad, this may allow for various interpretations of the oracle queries.
    \lean{QueryImpl}
    \uses{def:oracle_spec}
\end{definition}

\begin{definition}[Probabilistic Semantics of Oracle Computations]
    \label{def:probabilistic_semantics_of_oracle_computations}
    We can view oracle computations as probabilistic programs by considering what happens when 
    oracles respond uniformly at random. This gives rise to a probability distribution over possible 
    outputs (including the possibility of failure). The semantics maps each oracle query to a 
    uniform distribution over its possible responses.
    \lean{OracleComp.evalDist}
    \uses{def:oracle_computation}
\end{definition}

Once we have mapped an oracle computation to a probability distribution, we can define various associated probabilities, such as the probability of failure, or the probability of the output satisfying a given predicate (assuming it does not fail).

\begin{definition}[Simulating Oracle Queries with Other Oracles]
    \label{def:sim_oracle}
    We can simulate complex oracles using simpler ones by providing a translation mechanism. 
    A simulation oracle specifies how to implement queries in one specification using computations 
    in another specification, possibly maintaining additional state information during the simulation.
    % \lean{SimOracle.Stateful}
\end{definition}

\begin{definition}[Logging \& Caching Oracle Queries]
    \label{def:logging_caching_oracle_queries}
    Using the simulation framework, we can add logging and caching behaviors to oracle queries:
    \begin{itemize}
        \item Logging records all queries made during a computation
        \item Caching remembers query responses and reuses them for repeated queries
    \end{itemize}
    These are implemented as special cases of simulation oracles.
    \lean{loggingOracle, cachingOracle}
    \uses{def:oracle_computation, def:handling_oracle_queries}
\end{definition}

\begin{definition}[Random Oracle]
    \label{def:random_oracle}
    A random oracle is implemented as a caching oracle that uses lazy sampling:
    \begin{itemize}
        \item On first query: generates a uniform random response and caches it
        \item On repeated queries: returns the cached response
    \end{itemize}
    \lean{randomOracle}
    \uses{def:oracle_computation, def:handling_oracle_queries}
\end{definition}

% In summary, the VCVio library consists of:
% \begin{itemize}
%   \item A formal specification of oracles and their inputs/outputs (\texttt{OracleSpec}).
%   \item A monadic framework for computations that may perform oracle queries (\texttt{OracleComp}).
%   \item Effect handlers that permit the interpretation and simulation of oracle queries via
%         customizable implementations (\texttt{OracleImpl} and \texttt{SimOracle}).
%   \item Denotational semantics based on probability mass functions that allow the quantitative analysis
%         of such computations (\texttt{evalDist}, \texttt{probOutput}, \texttt{probFailure}, \texttt{probEvent}).
%   \item Extensions for logging, caching, and random oracles to support analysis of protocols using such oracles.
% \end{itemize}